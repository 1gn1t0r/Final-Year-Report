%!TEX root = ../Report.tex
%*******************************************************
% Abstract
%*******************************************************
%\renewcommand{\abstractname}{Abstract}
\pdfbookmark[1]{Abstract}{Abstract}
\begingroup
\let\clearpage\relax
\let\cleardoublepage\relax
\let\cleardoublepage\relax

\chapter*{Abstract}
Obtaining the required music licensing may not be viable for certain individuals and business owners. Alternatively the music can be composed and produced, however this requires time and skill.

For this project another alternative is investigated, to produce music using computer algorithms. 

The field of algorithmic music composition and is not new. The problem of algorithmically composing music is difficult and there is a wide range of research on the topic.

For this project, the relevent research is investigated. Possible algorithms were identified and implemented. Algorithms producing pleasant results include: Markov Models, a simple feed forward neural network, Markov Chains and Genetic Algorithms. These algorithms were kept and implemented in a end-user application. 

The Windows application developed features a friendly user interface that allows the user to compose, playback and store simple monophonic melodies in a certain style. In order to optimize the end user experience the algorithms were pretrained were possible.

Since it is difficult to quantify the pleasantness of a song, due to the inherent subjective experience, it was necessary to rely on pattern recognition and machine learning techniques. A weakness of the algorithms implemented is that they do not have a overarching theme or structure. 



\vfill

%\pdfbookmark[1]{Zusammenfassung}{Zusammenfassung}
%\chapter*{Zusammenfassung}
%Kurze Zusammenfassung des Inhaltes in deutscher Sprache\dots


\endgroup			

\vfill
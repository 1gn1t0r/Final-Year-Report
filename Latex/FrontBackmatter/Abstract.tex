%!TEX root = ../Report.tex
%*******************************************************
% Abstract
%*******************************************************
%\renewcommand{\abstractname}{Abstract}
\pdfbookmark[1]{Abstract}{Abstract}
\begingroup
\let\clearpage\relax
\let\cleardoublepage\relax
\let\cleardoublepage\relax

\chapter*{Abstract}
The field of algorithmic music composition is not new. The problem of algorithmically composing music is difficult and there is a wide range of research on the topic, however this field is inaccessible to most. 

This project covered a survey on algorithmic music composition and what it entails. In order to generate music in a certain style that is specified by the user, machine learning algorithms were used with the input dataset being one that corresponds to a certain style.

Various algorithms were investigated, designed and implemented. The best few were kept for inclusion in the developed friendly end-user application which exposes algorithmic music composition with machine learning to those who are not literate in the field. The application allows the user to select a musical category and generate a monophonic melody using a certain algorithms. The result can be visually displayed, played and stored.

Since it is difficult to quantify the pleasantness of a song, due to the inherent subjective experience, it was necessary to rely on statistical features and probalistic models. The algorithms investigated include: Hidden Markov Models, Markov Chains, Neural Networks, Stochastic Sampling and Genetic Algorithms. These algorithms are discussed and compared.

The weakness of the algorithms implemented is that they do not have an overarching theme or structure. 



\vfill

%\pdfbookmark[1]{Zusammenfassung}{Zusammenfassung}
%\chapter*{Zusammenfassung}
%Kurze Zusammenfassung des Inhaltes in deutscher Sprache\dots


\endgroup			

\vfill
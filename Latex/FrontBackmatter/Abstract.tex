%!TEX root = ../Report.tex
%*******************************************************
% Abstract
%*******************************************************
%\renewcommand{\abstractname}{Abstract}
\pdfbookmark[1]{Abstract}{Abstract}
\begingroup
\let\clearpage\relax
\let\cleardoublepage\relax
\let\cleardoublepage\relax

\chapter*{Abstract}

%TODO FEEEKS
Music is an art form of sound. For some commercial applications it may be impractical to produce music self or acquire the proper licensing. 

For this project an alternative composition method is investigated, generating music using computer algorithms. Specifically melodies are generated using learning algorithms for a specific style.

The field of algorithmic music composition is not new. The problem of algorithmically composing music is difficult and there is a wide range of research on the topic, however this field is inaccessible to most. 

For this project, the relevent research is investigated. Possible algorithms were identified and implemented. The algorithms investigated include: Hidden Markov Models, Markov Chains, Neural Networks, Stochastic Sampling and Genetic Algorithms. These algorithms are discussed and compared.

Feasible algorithms were kept for inclusion in a friendly user application. 
The Windows application developed features a friendly user interface that allows the user to compose, playback and store simple monophonic melodies in a certain style. 

Since it is difficult to quantify the pleasantness of a song, due to the inherent subjective experience, it was necessary to rely on statistical features and probalistic models. 

The weakness of the algorithms implemented is that they do not have an overarching theme or structure. 

\vfill

%\pdfbookmark[1]{Zusammenfassung}{Zusammenfassung}
%\chapter*{Zusammenfassung}
%Kurze Zusammenfassung des Inhaltes in deutscher Sprache\dots


\endgroup			

\vfill
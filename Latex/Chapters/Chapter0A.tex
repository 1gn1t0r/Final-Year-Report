%!TEX root = ../Report.tex
%********************************************************************
% Appendix
%*******************************************************
% If problems with the headers: get headings in appendix etc. right
%\markboth{\spacedlowsmallcaps{Appendix}}{\spacedlowsmallcaps{Appendix}}

\chapter{Algorithms} \label{chap:app_algo}
\section{Backpropogation}
For a feed forward network\footnote{Other topologies exist with more complex methods for obtaining the weights} with $n_\text{in}$ inputs, $n_\text{hidden}$ hidden units and $n_\text{out}$ output units the back-progogation algorithm works as follows
\begin{enumerate}
\item Initialize network with random weights
\item Repeat until algorithm terminates
\item 
\begin{itemize}
\item $\forall (\vec{x},\vec{t}) \in \text{training examples}$ do
\item 
\begin{enumerate}
\item Input instance $\vec{x}$ to network and compute $o_u\ \forall\ u\ \in \text{network}$
\item For each network output unit $k$ calculate error term\\
$\delta_k \leftarrow o_k (1-o_k)(t_k - o_k)$
\item For each hidden unit $h$ calculate the error term $\delta_h$\\
$\delta_h \leftarrow o_h (1- o_h) \sum_{k \in \text{outputs}} w_{kh}\delta_k$
\item Update each network weight $w_{ji}$\\
$w_{ji} \leftarrow w_{ji} + \eta \delta_j x_{ji}$
\end{enumerate}
\end{itemize}
\end{enumerate}
The complete derivation of the back-propogation algorithm for feed forward artificial neural networks can be found in Mitchell (2007) \cite{Mitchell1997}

\chapter{Platform Specifications}
\begin{table}[h!]
  \centering
  \caption{System specifications}
    \begin{tabular}{l|l}
    \toprule
    Operating System & Windows 8.1 \\
    Memory & 8GB 1333Mhz DDR3 \\
    CPU   & AMD X3 450 3.2GHZ 4 cores \\
    \bottomrule
    \end{tabular}%
  \label{tab:systemspecs}%
\end{table}%

\begin{table}[htbp]
  \centering
  \caption{Development specifications}
    \begin{tabular}{l|l}
    \toprule
    Language & C\# \\
    IDE   & Visual Studio 2015 Community \\
    .NET version & 4.5 \\
    \bottomrule
    \end{tabular}%
  \label{tab:languagespecs}%
\end{table}%


All tests were run with the specifications as given in table \ref{tab:systemspecs}. The development environment specifications are given in figure \ref{tab:languagespecs}.


\chapter{Source Code}
Listed here is the framework used to generate the results mentioned in this project.
\graffito{Code used to algorithmically compose music}

\section{DotNetMusic}
DotNetMusic contains the core music/\ac{MIDI} functionality and includes:
\begin{itemize}
\item Reading and Writing to \ac{MIDI} files using NAudio
\item Core data structures for music representation
\item Core playback functionality
\item \ac{WPF} control for displaying notes on a music sheet
\item Saving and loading of core music data structures using Google protocol buffers
\end{itemize}
DotNetMusic can be found at \href{https://github.com/stefan-j/DotNetMusic}{github.com/stefan-j/DotNetMusic}

\section{DotNetLearn}
Some additional machine learning implementation and useful math features.
\begin{itemize}
\item Implementation of Markov Model
\item Implementation of Hidden Markov Model with multiple observation Baum Welch Learning
\item Codification of data
\item Other statistical features
\end{itemize}
DotNetLearn can be found at \href{https://github.com/stefan-j/DotNetLearn}{github.com/stefan-j/DotNetLearn}

\section{GeneticMIDI}
GeneticMIDI contains the core functionality for algorithmically generating music.
\begin{itemize}
\item Generators for producing music algorithmically using Markov Models, Markov Chains, Neural Networks.
\item Genetic Algorithms (\ac{GP} tree structures)
\item Fitness functions for Genetic Algorithms including \ac{NCD}, Cosine Similarity (and frequency metrics)
\item Random generation of notes
\item The developed front-end visualizer application for producing music in a user-friendly manner
\end{itemize}

Libraries used include:
\begin{itemize}
\item A-Forge - For the neural networks and evolution of \acp{GA}.
\item NAudio - For reading, writing and playing of \ac{MIDI} data. 
\item IKVM - For the \ac{LTSM} network.
\item Proto.Net - .Net implementation of Google protocol buffers for saving of data structures
\end{itemize}

GeneticMIDI can be found at \href{https://github.com/stefan-j/GeneticMIDI}{github.com/stefan-j/GeneticMIDI}

